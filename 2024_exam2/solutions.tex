\documentclass{article}
\pagenumbering{gobble}
\usepackage[a3paper]{geometry}
\usepackage{amsmath, amssymb, amscd, amsthm, amsfonts}
\usepackage{graphicx}

\begin{document}

\section*{Question 1}

Let $F_{i,j}$ denote the gravitational force body $i$ exerts on body $j$.

Let $F_{centr} = M \omega^2 R = \frac{M \omega^2 a}{\sqrt{3}}$ denote the centripetal force on each of the three bodies.

We have

\begin{align}
    F_{centr} &= F_{21}\cos(30^\circ) +  F_{31}\cos(30^\circ) \\
    \frac{M \omega^2 a}{\sqrt{3}} &= 2 \frac{G M^2}{a^2} \frac{\sqrt{3}}{2} \\
    \omega &= \sqrt{\frac{3GM}{a^3}}
\end{align}

\section*{Question 2}

The previous question gives

\begin{equation}
    \omega^2 = \frac{3GM}{a^3} \implies a = \left(\frac{3GM}{\omega^2}\right)^{\frac{1}{3}}
\end{equation}

\section*{Question 3}

When the metal bar is not submerged we measure a mass $m _1= M$.

When the metal bar is completely submerged, we have a bouyance force $F_B = \rho_{\text{water}}Vg = M\frac{ \rho_{\text{water}}}{\rho_{\text{metal}}}g$.

The traction on the rope is $T = Mg - F_B = Mg\left(1 - \frac{ \rho_{\text{water}}}{\rho_{\text{metal}}}\right)$

So the mass measured here is $m_2 = M\left(1 - \frac{ \rho_{\text{water}}}{\rho_{\text{metal}}} \right)$.

We have the closed interval $[m_2, m_1] = [4 \text{ kg}, 5 \text{ kg}]$.

\section*{Question 4}

We have the same problem as question 3, we just need to change the numerical values of the parameters.

We have  $[m_2, m_1] = [4 \text{ kg}, 6 \text{ kg}]$

\end{document}